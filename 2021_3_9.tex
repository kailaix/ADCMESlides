%%%%%%%%%%%%%%%%%%%%%%%%%%%%%%%%%%%%%%%%%
% Beamer Presentation
% LaTeX Template
% Version 1.0 (10/11/12)
%
% This template has been downloaded from:
% http://www.LaTeXTemplates.com
%
% License:
% CC BY-NC-SA 3.0 (http://creativecommons.org/licenses/by-nc-sa/3.0/)
%
%%%%%%%%%%%%%%%%%%%%%%%%%%%%%%%%%%%%%%%%%

%----------------------------------------------------------------------------------------
%    PACKAGES AND THEMES
%----------------------------------------------------------------------------------------

\documentclass[usenames,dvipsnames]{beamer}
\usepackage{animate}
\usepackage{float}
\usepackage{bm}
\usepackage{mathtools}
\usepackage{extarrows}
\usepackage[utf8]{inputenc}
\usepackage[english]{babel}
\usepackage{minted}

\newcommand{\ChoL}{\mathsf{L}}
\newcommand{\bx}{\mathbf{x}}
\newcommand{\ii}{\mathrm{i}}
\newcommand{\bxi}{\bm{\xi}}
\newcommand{\bmu}{\bm{\mu}}
\newcommand{\bb}{\mathbf{b}}
\newcommand{\bA}{\mathbf{A}}
\newcommand{\bJ}{\mathbf{J}}
\newcommand{\bB}{\mathbf{B}}
\newcommand{\bM}{\mathbf{M}}

\newcommand{\by}{\mathbf{y}}
\newcommand{\bw}{\mathbf{w}}

\newcommand{\bX}{\mathbf{X}}
\newcommand{\bY}{\mathbf{Y}}
\newcommand{\bs}{\mathbf{s}}
\newcommand{\sign}{\mathrm{sign}}
\newcommand{\bt}[0]{\bm{\theta}}
\newcommand{\bc}{\mathbf{c}}
\newcommand{\bzero}{\mathbf{0}}
\renewcommand{\bf}{\mathbf{f}}
\newcommand{\bu}{\mathbf{u}}
\newcommand{\bv}[0]{\mathbf{v}}
\usepackage{minted}
\mode<presentation> {

% The Beamer class comes with a number of default slide themes
% which change the colors and layouts of slides. Below this is a list
% of all the themes, uncomment each in turn to see what they look like.

%\usetheme{default}
%\usetheme{AnnArbor}
%\usetheme{Antibes}
%\usetheme{Bergen}
%\usetheme{Berkeley}
%\usetheme{Berlin}
%\usetheme{Boadilla}
%\usetheme{CambridgeUS}
%\usetheme{Copenhagen}
%\usetheme{Darmstadt}
%\usetheme{Dresden}
%\usetheme{Frankfurt}
%\usetheme{Goettingen}
%\usetheme{Hannover}
%\usetheme{Ilmenau}
%\usetheme{JuanLesPins}
%\usetheme{Luebeck}
\usetheme{Madrid}
%\usetheme{Malmoe}
%\usetheme{Marburg}
%\usetheme{Montpellier}
%\usetheme{PaloAlto}
%\usetheme{Pittsburgh}
%\usetheme{Rochester}
%\usetheme{Singapore}
%\usetheme{Szeged}
%\usetheme{Warsaw}


% As well as themes, the Beamer class has a number of color themes
% for any slide theme. Uncomment each of these in turn to see how it
% changes the colors of your current slide theme.

%\usecolortheme{albatross}
\usecolortheme{beaver}
%\usecolortheme{beetle}
%\usecolortheme{crane}
%\usecolortheme{dolphin}
%\usecolortheme{dove}
%\usecolortheme{fly}
%\usecolortheme{lily}
%\usecolortheme{orchid}
%\usecolortheme{rose}
%\usecolortheme{seagull}
%\usecolortheme{seahorse}
%\usecolortheme{whale}
%\usecolortheme{wolverine}

%\setbeamertemplate{footline} % To remove the footer line in all slides uncomment this line
%\setbeamertemplate{footline}[page number] % To replace the footer line in all slides with a simple slide count uncomment this line

%\setbeamertemplate{navigation symbols}{} % To remove the navigation symbols from the bottom of all slides uncomment this line
}
\usepackage{booktabs}
\usepackage{makecell}
\usepackage{soul}
\newcommand{\red}[1]{\textcolor{red}{#1}}
%
%\usepackage{graphicx} % Allows including images
%\usepackage{booktabs} % Allows the use of \toprule, \midrule and \bottomrule in tables
%
%
%\usepackage{amsthm}
%
%\usepackage{todonotes}
%\usepackage{floatrow}
%
%\usepackage{pgfplots,algorithmic,algorithm}
\usepackage{algorithmicx}
\usepackage{algpseudocode}
%\usepackage[toc,page]{appendix}
%\usepackage{float}
%\usepackage{booktabs}
%\usepackage{bm}
%
%\theoremstyle{definition}
%
\newcommand{\RR}[0]{\mathbb{R}}
%
%\newcommand{\bx}{\mathbf{x}}
%\newcommand{\ii}{\mathrm{i}}
%\newcommand{\bxi}{\bm{\xi}}
%\newcommand{\bmu}{\bm{\mu}}
%\newcommand{\bb}{\mathbf{b}}
%\newcommand{\bA}{\mathbf{A}}
%\newcommand{\bJ}{\mathbf{J}}
%\newcommand{\bB}{\mathbf{B}}
%\newcommand{\bM}{\mathbf{M}}
%\newcommand{\bF}{\mathbf{F}}
%
%\newcommand{\by}{\mathbf{y}}
%\newcommand{\bw}{\mathbf{w}}
%\newcommand{\bn}{\mathbf{n}}
%
%\newcommand{\bX}{\mathbf{X}}
%\newcommand{\bY}{\mathbf{Y}}
%\newcommand{\bs}{\mathbf{s}}
%\newcommand{\sign}{\mathrm{sign}}
%\newcommand{\bt}[0]{\bm{\theta}}
%\newcommand{\bc}{\mathbf{c}}
%\newcommand{\bzero}{\mathbf{0}}
%\renewcommand{\bf}{\mathbf{f}}
%\newcommand{\bu}{\mathbf{u}}
%\newcommand{\bv}[0]{\mathbf{v}}

\AtBeginSection[]
{
   \begin{frame}
       \frametitle{Outline}
       \tableofcontents[currentsection]
   \end{frame}
}

%----------------------------------------------------------------------------------------
%    TITLE PAGE
%----------------------------------------------------------------------------------------
\usepackage{bm}
\newcommand*{\TakeFourierOrnament}[1]{{%
\fontencoding{U}\fontfamily{futs}\selectfont\char#1}}
\newcommand*{\danger}{\TakeFourierOrnament{66}}

\title[Distributed ML for Scientific Computing]{ADCME MPI: Distributed Machine Learning for Computational Engineering} % The short title appears at the bottom of every slide, the full title is only on the title page

\author[ADCME-MPI]{Kailai Xu and Eric Darve \\ \url{https://github.com/kailaix/ADCME.jl}} % Your name
%\institute[] % Your institution as it will appear on the bottom of every slide, may be shorthand to save space
%{
%%ICME, Stanford University \\ % Your institution for the title page
%%\medskip
%%\textit{kailaix@stanford.edu}\quad \textit{darve@stanford.edu} % Your email address
%}
\date{}% Date, can be changed to a custom date
% Mathematics of PDEs


\begin{document}

\usebackgroundtemplate{%
\begin{picture}(0,250)
\centering
	{{\includegraphics[width=1.0\paperwidth]{figures/background}}}
\end{picture}
  } 
%\usebackgroundtemplate{%
%  \includegraphics[width=\paperwidth,height=\paperheight]{figures/back}} 
\begin{frame}

\titlepage % Print the title page as the first slide

%dfa
\end{frame}
\usebackgroundtemplate{}

\section{Inverse Modeling}




\begin{frame}
	\frametitle{Inverse Modeling}
	\begin{figure}
		\centering
		\includegraphics[width=1.0\textwidth]{figures/inverse3}
	\end{figure}
\end{frame}



\begin{frame}
	\frametitle{Inverse Modeling}
	We can formulate inverse modeling as a PDE-constrained optimization problem 
	\begin{equation*}
		\min_{\theta} L_h(u_h) \quad \mathrm{s.t.}\; F_h(\theta, u_h) = 0
	\end{equation*}
	\begin{itemize}
		\item The \textcolor{red}{loss function} $L_h$ measures the discrepancy between the prediction $u_h$ and the observation $u_{\mathrm{obs}}$, e.g., $L_h(u_h) = \|u_h - u_{\mathrm{obs}}\|_2^2$. 
		\item $\theta$ is the \textcolor{red}{model parameter} to be calibrated. 
		\item The \textcolor{red}{physics constraints} $F_h(\theta, u_h)=0$ are described by a system of partial differential equations or differential algebraic equations (DAEs); e.g., 
		$$F_h(\theta, u_h) = \mathbf{A}(\theta) u_h - f_h = 0$$
	\end{itemize}
\end{frame}




\begin{frame}
	\frametitle{Function Inverse Problem}
	
	\begin{equation*}
		\min_{\textcolor{red}{f}} L_h(u_h) \quad \mathrm{s.t.}\; F_h(\textcolor{red}{f}, u_h) = 0
	\end{equation*}
	
	What if the unknown is a \textcolor{red}{function} instead of a set of parameters?
\begin{itemize}
	\item Koopman operator in dynamical systems.
	\item Constitutive relations in solid mechanics. 
	\item Turbulent closure relations in fluid mechanics.
	\item ...
\end{itemize}

The candidate solution space is \textcolor{red}{infinite dimensional}.

\end{frame}


\begin{frame}
	\frametitle{Machine Learning for Computational Engineering}
	$$\min_{\theta} L_h(u_h) \quad \mathrm{s.t.}\;\boxed{F_h(\textcolor{red}{NN_\theta}, u_h) = 0} \leftarrow \mbox{Solved numerically}$$
	\vspace{-0.5cm}
	\begin{enumerate}
		\item Use a deep neural network to approximate the (high dimensional) unknown function;
		\item Solve $u_h$ from the physical constraint using a numerical PDE solver;
		\item Apply an unconstrained optimizer to the reduced problem
		$$\min_{\theta} L_h(\textcolor{red}{u_h(\theta)})$$
	\end{enumerate}
\vspace{-0.3cm}
	\begin{figure}[hbt]
  \includegraphics[width=0.75\textwidth]{figures/physics_based_machine_learning.png}
\end{figure}
\end{frame}



\begin{frame}
	\frametitle{Gradient Based Optimization}
	\begin{equation*}
		\min_{\theta} L_h(u_h) \quad \mathrm{s.t.}\; F_h(\theta, u_h) = 0 \qquad\Leftrightarrow\qquad \min_{\theta} L_h({u_h(\theta)})
		\end{equation*}
	\vspace{-0.5cm}
	\begin{itemize}
		 \item We can now apply a gradient-based optimization method if we can \textcolor{red}{calculate a descent direction} $g^k$
		$$\theta^{k+1} \gets \theta^k - \alpha g^k$$ 
	\end{itemize}

	\begin{figure}[hbt]
	\centering
  \includegraphics[width=0.6\textwidth]{figures/im.pdf}
\end{figure}

\end{frame}



\section{Automatic Differentiation}


\begin{frame}
	\frametitle{Automatic Differentiation}
	The fact that bridges the \textcolor{red}{technical} gap between machine learning and inverse modeling:
	\begin{itemize}
		\item Deep learning (and many other machine learning techniques) and numerical schemes share the same computational model: composition of individual operators. 
	\end{itemize}
	
	
	\begin{minipage}[t]{0.4\textwidth}
		
		\
		
		
		
		\begin{center}
			\textcolor{red}{Mathematical Fact}
			
			\
			
			Back-propagation 
			
			$||$
			
			Reverse-mode
			
			Automatic Differentiation 
			
			$||$
			
			Discrete 
			
			Adjoint-State Method
		\end{center}
	\end{minipage}~
	\begin{minipage}[t]{0.6\textwidth}
		\begin{figure}[hbt]
			\includegraphics[width=0.8\textwidth]{figures/compare-NN-PDE.png}
		\end{figure}
	\end{minipage}
	
\end{frame}

\begin{frame}
	\frametitle{Computational Graph for Numerical Schemes}
	
	\begin{itemize}
		\item To leverage automatic differentiation for inverse modeling, we need to express the numerical schemes in the ``AD language'': computational graph. 
		\item No matter how complicated a numerical scheme is, it can be decomposed into a collection of operators that are interlinked via state variable dependencies. 
	\end{itemize}
	
	\begin{figure}[hbt]
		\includegraphics[width=1.0\textwidth]{figures/cgnum}
	\end{figure}
	
	
	
\end{frame}




\begin{frame}
	\frametitle{ADCME: Computational-Graph-based Numerical Simulation}
	
	\begin{figure}[hbt]
		\includegraphics[width=1.0\textwidth]{figures/custom}
	\end{figure}
\end{frame}

\begin{frame}
	\frametitle{How ADCME works}
	\begin{itemize}
		\item ADCME translates your \textcolor{red}{high level} numerical simulation codes to computational graph and then the computations are delegated to a heterogeneous task-based parallel computing environment through TensorFlow runtime. 
	\end{itemize}
	\begin{figure}[hbt]
		\includegraphics[width=1.0\textwidth]{figures/routine2.png}
	\end{figure}
\end{frame}



\section{Distributed Computing for Computational Engineering}


\begin{frame}
\frametitle{Common Distributed Computing Patterns in DL}

$$L(\theta) = \sum_{i=1}^N (NN(x_i; \theta) - y_i)^2$$

\begin{figure}
	\centering 
		\includegraphics[width=0.48\textwidth]{figures/mpi/data_parallel}~
	\includegraphics[width=0.48\textwidth]{figures/mpi/parameter_server}
\end{figure}

\end{frame}



\begin{frame}
\frametitle{Distributed Computing in ML for Computational Engineering}
\begin{minipage}{.5\textwidth}
	Consider a time-dependent PDE, where the state variable 
	$$u_k = [u_k^{(1)}\ u_k^{(2)}\ \cdots u_k^{(P)}]$$
	is stored on $P$ machines. Each time step requires a distributed numerical solver. 
	\begin{equation*}
\begin{aligned}
\min_{\theta} &\; L(u_n)\\ 
\text{s.t.} &\; A(\theta)u_2 = h(u_1; \theta) + g\\ 
&\; A(\theta)u_3 = h(u_2; \theta) + g\\
&\;\vdots \\ 
&\; A(\theta)u_n = h(u_{n-1}; \theta) + g
\end{aligned}
	\end{equation*}
\end{minipage}~
\begin{minipage}{.5\textwidth}
	\includegraphics[width=0.8\textwidth]{figures/mpi/ourmodel}
\end{minipage}
\end{frame}


\begin{frame}
\frametitle{ADCME-MPI}

ADCME-MPI abstracts distributed computing as a node in the computational graph. The ADCME-MPI model is \textbf{transparent}.
\begin{itemize}
\item ADCME takes responsibility for MPI communication and gradient back-propagation across clusters;
\item users can adapt their single processor codes to a distributed computing environment with little efforts.
\end{itemize}
 


\begin{figure}
	\centering 
	\includegraphics[width=0.49\textwidth]{figures/mpi/intro1}
	\includegraphics[width=0.49\textwidth]{figures/mpi/intro2}
\end{figure}


\end{frame}


\begin{frame}[fragile]{Example}

\begin{minipage}{.5\textwidth}
	Consider a simple function:
	$$L(\theta) = 1 + \theta + \theta^2 + \theta^3$$
		\includegraphics[width=1.0\textwidth]{figures/mpi/simple}
\end{minipage}~
\begin{minipage}{.5\textwidth}
	\begin{minted}{julia}
using ADCME
mpi_init() # initialize MPI 
theta0 = placeholder(1.0)
theta = mpi_bcast(theta0)
l = theta^mpi_rank()
L = mpi_sum(l)
g = gradients(L, theta0)
# initialize a Session
sess = Session(); init(sess)
L_value = run(sess, L) 
g_value = run(sess, g)
mpi_finalize() # finalize MPI 
	\end{minted}
\end{minipage}


\end{frame}


\begin{frame}{Distributed Optimization}
	
	In the ADCME-MPI, we can convert a serial optimizer to a distributed optimizer by inserting some communication codes:
\begin{figure}
\centering 
\includegraphics[width=0.8\textwidth]{figures/mpi/optimization}
\end{figure}
\end{frame}

\begin{frame}{Benchmarks}
	\begin{minipage}[c]{.52\textwidth}
		\begin{equation*}
			\begin{aligned}
				\min_{\theta}&\; J(\theta):=\sum_{i\in \mathcal{I}} \left( u(\bx_i) - u_i \right)^2\\ 
				\text{s.t.} &\; \nabla \cdot (\textcolor{red}{NN_\theta(\bx)} \nabla u(\bx))  = f(\bx),\ \bx \in \Omega \\ 
				&\; u(\bx) = 0,\  \bx \in \partial \Omega
			\end{aligned}
		\end{equation*}
	\begin{figure}
		\centering
			\includegraphics[width=0.45\textwidth]{figures/mpi/grid}~
				\includegraphics[width=0.45\textwidth]{figures/mpi/hypre}
	\end{figure}

	\end{minipage}~
	\begin{minipage}[c]{.48\textwidth}
		\includegraphics[width=1.0\textwidth]{figures/mpi/poisson}
	\end{minipage}
	
	
\end{frame}
\begin{frame}
	\begin{figure}[htpb]
		\centering 		
		\includegraphics[width=1.0\textwidth]{figures/mpi/overhead_adcme}
	\end{figure}
\end{frame}


\begin{frame}{Reference}

For more technical details, benchmarks, or use cases:
\begin{itemize}
\item AAAI Conference Paper: \textit{ADCME MPI: Distributed Machine Learning
for Computational Engineering}
\item Full paper: \textit{Distributed Machine Learning for Computational Engineering using MPI
}

 \url{https://arxiv.org/pdf/2011.01349.pdf}
\item Software documentation: \url{https://kailaix.github.io/ADCME.jl/dev/}
\end{itemize}
\end{frame}

\begin{frame}
	\frametitle{A General Approach to Inverse Modeling}
	\begin{figure}[hbt]
		\includegraphics[width=1.0\textwidth]{figures/summary.png}
	\end{figure}
\end{frame}



%}
%\usebackgroundtemplate{}
%----------------------------------------------------------------------------------------
%    PRESENTATION SLIDES
%----------------------------------------------------------------------------------------

%------------------------------------------------



\end{document} 